\documentclass[a4paper]{article}

\usepackage[utf8]{inputenc}
\usepackage[T2A]{fontenc}
\usepackage[english,russian]{babel}
\usepackage{float}
\usepackage{array}

\title{Маршрутизация в компьютерной сети с применением глубокого обучения}
\author{Панчишин Иван Романович, группа М41381с}
\date{2021-06-01}

\begin{document}

\maketitle

\section{Описание работы}

В работе сравниваются 3 алгоритма децентрализованной маршрутизации:
link-state, Q и DQN.

В link-state каждый узел сети строит карту сети и выполняет поиск кратчайшего
пути при помощи алгоритма Дейкстры.

В Q применяется обучение с подкреплением, где каждый маршрутизатор поддерживает
таблицу с оценками стоимости пути до каждого узла.

DQN основывается на алгоритме Q и использует нейронную сеть.

Алгоритмы будут рассматриваться в рамках компьютерной сети, но их можно
адаптировать под использование в конвейерных сетях.

\section{Участники}

\begin{enumerate}
    \item Панчишин И.Р. (М41381с).
\end{enumerate}

\section{План исследования}

\begin{table}[H]
    \centering
    \begin{tabular}{cp{9cm}p{2cm}}
        № &
        Описание задачи &
        Плановый срок выполнения \\
        \hline
        1 &
        Реализовать имитационную модель абстрактной компьютерной сети &
        2021-06-08 \\
        2 &
        Реализовать алгоритм маршрутизации link-state &
        2021-06-09 \\
        3 &
        Реализовать алгоритм маршрутизации Q &
        2021-06-10 \\
        4 &
        Сравнить работу алгоритмов link-state и Q и написать отчет по
        промежуточному этапу сдачи курсового проекта &
        2021-06-11 \\
        5 &
        Реализовать алгоритм маршрутизации DQN &
        2021-06-15 \\
        6 &
        Реализовать метод получения графовых эмбеддингов Laplacian Eigenmaps &
        2021-06-17 \\
        7 &
        Сравнить обобщающую способность модели DQN на различных
        эмбеддингах &
        2021-06-20 \\
        8 &
        Сравнить работу алгоритма DQN с алгоритмами, где не используется
        нейронная сеть &
        2021-06-22 \\
        \hline
    \end{tabular}
\end{table}

\end{document}
